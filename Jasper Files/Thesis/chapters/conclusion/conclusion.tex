%!TeX root=../../main.tex
\chapter{Conclusion}                                 \label{ch:conclusion}

%======================================
\section{Overview} \label{sec:overview}
In this thesis we reseached the Kubernetes technology stack regarding network security and defined the possibility af conflicts rising within due to misconfigurations between different layers of the stack. After we defined our problem statement we researched existing state-of-the art solutions regarding these conflicts and identified a knowledge gap within the current solutions as well as a possibility for improvement for Kano, an existing research solution. We continued by leveraging Kano's ideas to implement our algorithm that detects and reports on these conflicts. The implementation uses a incremental approach of updating the last locally saved cluster state, triggered by the monitoring and capturing of events that influence the connections between containers. 
\\[10pt]

In our evaluation we compared the incremental update functionality of our implementation against Kano's generative approach and concluded that the generative approach is faster than our incrementel solution for smaller clusters, whereas our incremental approach outperforms Kano's when the cluster becomes sufficiently large. A drawback of our incremental approach is the increased memory usage due to the storage of cluster state information as well as extra data required for the conflict detection \todo{Also say this in evaluation chapter}. In our second experiment we evaluated the general time and memory consumption for our entire conflict detection algorithm \todo{finish this}
\\[10pt]

%======================================
\section{footnote} \label{sec:footnote}
\todo{Different section name}
%Kano paper 2
%gitlab reference?

%======================================
\section{Future work} \label{sec:futurework}
We briefly mentioned a startup-verification method to detect already existing conflicts within the cluster, which is implemented but remains untested due to time constraints for this thesis. When the memory consumption of the implemented algorithm proves to be too excessive the generative approach combined with the conflict detection could possibly be executed periodically to confirm the conflict-free state of the cluster.
\\[10pt]

The solution algorithm does not monitor cloud level changes to the cluster but instead mimicks it by generating randomised Openstack security groups and their rules. Future work could include the continuous monitoring for events that influence inter-VM connections similarly to the implemented cluster-layer watcher, upon the capturing of which conflict detection gets triggered as well.
\\[10pt]



\cleardoublepage
