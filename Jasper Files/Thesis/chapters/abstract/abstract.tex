%!TeX root=../../main.tex
\chapter{Abstract} \label{ch:abstract}
Cloud-native is an approach to building and deploying applications within containers in cloud environments and has become an industry standard thanks to its many advantages such as cost efficiency, scalability, and automation. Many tools exist throughout the different layers of the technology stack to help facilitate the cloud-native approach, such as Docker \cite{docker}, Kubernetes \cite{Bernstein2014} and OpenStack \cite{Openstack}, all of which have their own, albeit sometimes overlapping, functionalities. An important aspect regarding the security of these cloud infrastructures is the management of network communication rules throughout the stack, which can help prevent malicious attacks from spreading throughout the stack by restricting communication between components. The dynamic nature of container clusters makes this a non-trivial task.
\\[10pt]

In this thesis we will dive into OpenStack \cite{Openstack} and Kubernetes \cite{Bernstein2014}, two tools with their own network security rule functionalities, respectively situated in the cloud and container cluster layer of the cloud-native technology stack. OpenStack provides a way to orchestrate virtual machines in a cluster and offers security group rules as a way to specify which of these instances are allowed to communicate  \cite{sgrule}. Kubernetes on the other hand deploys containerized applications on these Openstack instances and offers network policies as a way to achieve communication regulation between containers \cite{nps}. If these network security rules in the different layers of the technology stack are not properly aligned conflict may arise which can introduce new attack vectors or prohibit the correct working of clusters. We found a gap in the current state-of-the-art research solution, as the conflict detection for these types of configuration conflicts throughout different stack layers has not been researched to the best of our knowledge.
\\[10pt]

This thesis presents a conflict detection algorithm to verify network security rules between the cloud and cluster layer, specifically Kubernetes network policies and OpenStack security group rules. To do this, it leverages the reachability matrix introduced in the research paper of Kano \cite{kano}, while trying to increase time performance by including an incremental update approach for this reachability matrix. The conflict detection is triggered by any event that can influence the connections between containers in the cluster, found by continuously monitoring the cluster. When such an event is captured our incremental approach is used to find any connectivity changes which are then verified against a mocked cloud layer of security group rules to find any newly introduced conflicts. We evaluate the proposed algorithm and compare our incremental approach to an existing generative approach of updating the reachability matrix. The results show that our incremental update approach proves to be faster when the cluster has increased enough in size with the drawback of extra memory consumption. Our entire conflict detection solution has proven to only add a little overhead in time on top of the incremental update approach. With a maximum average of 474ms in the biggest cluster size of our experiments, it has proven to be faster than a cold-start pod deployment.
\cleardoublepage
