%!TeX root=../../main.tex
\chapter{Abstract} \label{ch:abstract}
The cloud native approach of developing and deploying containerized applications in cloud environments is not a new phenomenom, and as time passed the expectations for these type of systems has only grown. In order to tackle these demands many tools have been introduced into the cloud native ecosystem that support human operators in handling clusters of containerized applications. These tools are situated in many different, sometimes overlapping, layers of the technology stack and often offer similar functionalities. An important one of these functions is managing network connectivity between components in order to decrease network congestion and to contain the spread of a possible attack by malicious actors. \\[10pt]

In this thesis we will dive into Openstack and Kubernetes, two tools with their own network security rule functionalities situated in the cloud and cluster layer of the cloud native stack respectively. Openstack provides a way to orchestrate virtual machines in a cluster and offers security group rules as a way to specify which of these instances are allowed to communicate. Kubernetes on the other hand deploys containerized applications on these Openstack instances and offers network policies as a way to achieve communication regulation between containers. If these network security rules in the different layers of the technology stack are not properly aligned they can introduce new attack vectors or prohibit the correct working of clusters. Due to the fluctuating nature of Kubernetes clusters the chance of accidental misconfiguration and therefore conflicts increases as the cluster grows in size.
\\[10pt]

This thesis presents a conflict detection algorithm to verify network security rules between the cloud and cluster layer, specifically Kubernetes and Openstack. To do this it leverages on the concept of the reachabilitymatrix introduced in the research paper of Kano \cite{kano}, while trying to increase time performance by including an incremental update approach for this reachabilitymatrix. The conflict detection triggers are found by continously monitoring and capturing events on the cluster and filtering out those that can influence the connections between containers. The incremental approach is used to find any connectiviy changes which are then verfied against a mocked cloud layer of security group rules to find any newly introduced conflicts. We evaluate the proposed algorithm and compare our incremental approach to an existing generative approach for the reachabilitymatrix. The results show that our incremental approach proves to be faster when the cluster has increased enough in size with the drawback of extra memory consumption. 
\cleardoublepage
