%!TeX root=../../main.tex
\chapter{Samenvatting}                                 \label{ch:abstractNL}

cloud-native ontwikkeling is een methode om container-applicaties te implementeren en uit te rollen in cloud-omgevingen in plaats van op on-site infrastructuur, en is een standaard geworden in de industrie dankzij de vele voordelen zoals kost-efficiëntie, schaalbaarheid en automatisering. De cloud-native technologiestapel kan opgedeeld worden in verschillende lagen met hun eigen verantwoordelijkheden, zoals de code, container en cloud laag. Bijvoorbeeld, de container laag behandelt het uitrollen en onderhouden van container applicaties, terwijl de cloud laag alles opvolgt in betrekking tot de virtuele machines waarop de container applicaties draaien.
\\[10pt]

Deze verschillende lagen zijn vaak beheerd door gespecialiseerde tools met hun eigen veiligheidsfunctionaliteiten. Een vaak wederkerende functionaliteit is de netwerkcommunicatieregels dat de communicatie tussen verschillende onderdelen beperkt om zodoende de verspreiding van malafide aanvallen te voorkomen. Desondanks kunnen conflicten nog altijd ontstaan wanneer de netwerkcommunicatieregels van de verschillende lagen niet op elkaar zijn afgestemd, wat resulteert in een onbereikbaar component of een onverwachte opening voor cyber-aanvallen. Bijkomend wordt het afstemmen van deze regels bemoeilijkt door de dynamische natuur van cloud deployments doordat componenten toegevoegd of verwijderd kunnen worden door de automatische schaalveranderingen.
\\[10pt]

In deze masterproef presenteren we een conflictdetectie-algoritme om netwerkbeveiligingsregels tussen de cloud- en clusterlaag, met name Kubernetes en Openstack, te verifiëren. Om dit te doen maken we gebruik van de reachabilitymatrix, geïntroduceerd in de publicatie van Kano. Tegelijkertijd proberen we de tijdprestaties van Kano te verbeteren door een incrementele update methode voor deze reachabilitymatrix te implementeren. Conflict detectie word getriggerd door de cluster continu te monitoren en de events te filteren die de verbindingen tussen containers in de cluster kunnen beïnvloeden. Wanneer zo een event wordt gevonden wordt de incrementele update methode gebruikt om eventuele connectiviteitswijzigingen te vinden, die vervolgens worden vergeleken met een nagebootste cloudlaag van security group rules om nieuw geïntroduceerde conflicten te vinden. We evalueren het voorgestelde algoritme en vergelijken onze incrementele update methode met de bestaande Kano methode. De resultaten laten zien dat onze implementatie sneller blijkt te zijn dan de kano's zodra de cluster een bepaalde grootte heeft behaald, met als nadeel extra geheugenverbruik. Met een maximum gemiddelde van 474ms in de grootste cluster setup van ons experiment bewijst onze conflict detectie methode sneller te zijn dan de gemiddelde cold-start container opstart tijd, terwijl hetslechts een kleine extra tijdskost bovenop de incrementele update methode introduceert.
