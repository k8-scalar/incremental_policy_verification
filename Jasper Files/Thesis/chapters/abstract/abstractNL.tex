%!TeX root=../../main.tex
\chapter{Samenvatting}                                 \label{ch:abstractNL}

De cloud-native benadering van het ontwikkelen en implementeren van container-applicaties in cloud-omgevingen is geen nieuw fenomeen, en naarmate de tijd verstrijkt, zijn de verwachtingen voor dit soort systemen alleen maar gegroeid. Om aan deze eisen te voldoen zijn er meerdere tools geïntroduceerd in het cloud-native ecosysteem die menselijke operatoren kunnen ondersteunen in het beheer van deze container-applicatie clusters. Deze tools bevinden zich in veel verschillende, soms overlappende, lagen van de technologiestapel en bieden vaak vergelijkbare functionaliteiten. Een van deze functies is het beheren van de netwerkconnectiviteit tussen componenten om netwerkcongestie te verminderen en de verspreiding van een mogelijke aanval door kwaadwillende actoren tegen te gaan.\\[10pt]


In deze masterproef nemen we een duik in de specificaties van Openstack en Kubernetes, twee tools met hun eigen netwerkbeveiligingsfuncties die zich respectievelijk in de cloud- en clusterlaag van de cloud-native stack bevinden. Openstack dient voornamelijk om virtuele machines in een cluster te orchestreren en biedt security group rules aan als manier om te specificeren welke van deze instanties mogen communiceren met elkaar. Kubernetes daarentegen implementeert gecontaineriseerde applicaties bovenop deze OpenStack instanties en biedt network policies aan als methode om communicatie tussen containers te managen. Als deze netwerkbeveiligingsregels in de verschillende lagen van de technologiestapel niet goed op elkaar zijn afgestemd, kunnen ze nieuwe aanvalsvectoren introduceren of de correcte werking van clusters tegenwerken. Door de fluctuerende aard van Kubernetes clusters neemt de kans op onbedoelde misconfiguratie en daaropvolgende conflicten toe naarmate de cluster groter wordt.
\\[10pt]


Deze masterproef presenteert een conflictdetectie-algoritme om netwerkbeveiligingsregels tussen de cloud- en clusterlaag, met name Kubernetes en Openstack, te verifiëren. Om dit te doen maakt het gebruik van het concept van de reachabilitymatrix geïntroduceerd in de onderzoeksoplossing van Kano \cite{kano}. Tegelijkertijd wordt geprobeerd de tijdprestaties te verbeteren door een incrementele update-aanpak voor deze reachabilitymatrix te implementeren. Zo een update word getriggerd door continu de cluster te monitoren en gebeurtenissen uit te filteren die de verbindingen tussen containers kunnen beïnvloeden. De incrementele update wordt dan gebruikt om eventuele connectiviteitswijzigingen te vinden, die vervolgens worden vergeleken met een nagebootste cloudlaag van security group rules om nieuw geïntroduceerde conflicten te vinden. We evalueren het voorgestelde algoritme en vergelijken onze incrementele aanpak met de bestaande generatieve aanpak voor het creëren van de  reachabilitymatrix. De resultaten laten zien dat onze incrementele aanpak om de reachabilitymatrix te updaten sneller blijkt te zijn dan de deze volledig te regenereren zodra de cluster voldoende groot is geworden, met het nadeel extra geheugen te verbruiken. 



